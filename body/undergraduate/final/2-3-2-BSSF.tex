\subsection{$2m$阶平滑样条拟合—— Bform }
\label{sec:FINAL2mSmoothingSplinesFitting}

\begin{defn}[$2m$阶平滑样条拟合]
  \label{def:FINAL2mSmoothingSplinesFitting}
  给定一组严格递增的数据点$\mathbf{x}=\{x_{i}\}_{i=1}^{N}$
  和对应的误差权重$\mathbf{w}=\{w_{i}\}_{i=1}^{N}$。记
  \begin{equation}
    \label{eq:FINALSSFBFormSpace}
      X_{\textnormal{BSSF}}:=\{g\in C^{m-1}([x_{1},x_{N}])\cap
      D^{m}([x_{1},x_{N}]\setminus \{x_{i}\}_{i=2}^{N-1}):
  \ \lambda D^{m}g\in \mathbb{S}_{m-1}^{m-2}(\mathbf{x})\},
  \end{equation}
  其中$\lambda(t)$为粗糙度权重
  (见定义~\ref{def:FINALcubicSmoothingSplinesFitting}) 。
  给定容忍度(tolerance)$S\ge 0$
  以及一组给定的值$\mathbf{y}=\{y_{i}\}_{i=1}^{N}$,
  称样条函数$f\in X_{\textnormal{BSSF}}$是$\{(x_{i},y_{i})\}_{i=1}^{N}$
  在$ X_{\textnormal{BSSF}}$下误差度量不超过$S$的$2m$阶平滑样条拟合,
  如果
  \begin{equation}
    \label{eq:FINAL2mSmoothingSplinesFitting}
    f = \arg \min_{g\in  X_{\textnormal{BSSF}}\textnormal{ with } E(g)\le S} F(D^{m}g),
  \end{equation}
  其中$E(g),\ F(D^{m}g)$分别如
  (\ref{eq:FINALerrorMeasure}), (\ref{eq:FINALroughnessMeasure}) 中定义。
\end{defn}

\begin{thm}
  \label{thm:FINAL2mSmoothingSplinesFittingEquiv}
  定义 \ref{def:FINAL2mSmoothingSplinesFitting} 中的问题与以下问题等价:
  寻找$f\in  X_{\textnormal{BSSF}}$以及平滑参数$\rho\in \mathbb{R}$,使得$E(f)=S$且
  \begin{equation}
    \label{eq:FINAL2mSmoothingSplinesFittingEquiv}
    f = \arg \min_{g\in  X_{\textnormal{BSSF}}} \left( \rho E(g)+F(D^{m}g) \right).
  \end{equation}
\end{thm}
\begin{proof}
  欲求~(\ref{eq:FINAL2mSmoothingSplinesFitting}),我们注意到有约束条件
  $E(g)\le S$,可以引入一个辅助变量$z\in \mathbb{R}$,以及
  一个Lagrange乘数 (或KKT乘数) $\rho\ge0$,使得我们只需考虑
  以下最值问题:
  \begin{displaymath}
    \min_{g\in  X_{\textnormal{BSSF}},z\in \mathbb{R},\rho\ge 0}G(g,z,\rho),
  \end{displaymath}
  其中
  \begin{displaymath}
    G(g,z,\rho):=\rho(E(g)+z^{2}-S)+F(D^{m}g).
  \end{displaymath}
  并且由 Lagrange 条件,$G_{\rho}=G_{z}=0$,即
  \begin{displaymath}
    \begin{aligned}
      E(g)&=S-z^{2}, \\
      \rho z&=0.
    \end{aligned}
  \end{displaymath}
  因此,若已求出$\rho$的值,那么只需寻找$f\in X_{\textnormal{BSSF}}$,使得
  (\ref{eq:FINAL2mSmoothingSplinesFittingEquiv})成立。
  求解$\rho$的过程比较复杂,我们在证明完$f$的存在唯一性后讲述其具体细节。
\end{proof}

将~(\ref{eq:FINALSSFBFormSpace})和
(\ref{eq:FINALSSFppFormSpace}) 做对比知,当 $m=2$时,空间
$X_{\textnormal{BSSF}}$和$X_{\textnormal{ppSSF}}$(在样条格式相同的意义下)是一样的;
将~(\ref{eq:FINAL2mSmoothingSplinesFittingEquiv}) 和
(\ref{eq:FINALcubicSmoothingSpline}) 做对比知,二者所要解决的问题的形式也类似
(令后者的$p=\rho/(\rho+1)$)。
然而,由于样条格式不同,且 B 样条要求阶数为任意偶数,所以在这一节解决问题
的方法,会和 \ref{sec:FINALcubicSmoothingSplineFitting} 节截然不同。
在证明之前,我们也给出一个类似于引理 \ref{lem:FINALminimumBendingEnergy}
的一个引理,用于验证目标函数$f$具有自然边界性质,即
\begin{equation}
  \label{eq:FINALnaturalOrder2m}
  \forall j=m,\cdots,2m-2,\quad
    D^{j}f(x_{1})=D^{j}f(x_{N})=0.
  \end{equation}

\begin{lem}
  \label{lem:FINALminimumBendingEnergyForBSpline}
  对于任意
  满足条件$g(x_{i})=y_{i},\ i=1,\cdots,N$的$g\in C^{m-1}([x_{1},x_{N}])\cap
      D^{m}([x_{1},x_{N}]\setminus \{x_{i}\}_{i=2}^{N-1})$,
  通过对数据点
  $\{(x_{i},y_{i})\}_{i=1}^{N}$插值得到的自然样条$f\in X_{\textnormal{BSSF}}$
  满足
  \begin{equation}
    \label{eq:FINALminimumBendingEnergyForBSpline}
    F(D^{m}f)\le F(D^{m}g),
  \end{equation}
  其中$F(D^{m}f),\ F(D^{m}g)$如~(\ref{eq:FINALroughnessMeasure})中定义。
  上式等号成立当且仅当$g=f$。
\end{lem}
\begin{proof}
  与引理 \ref{lem:FINALminimumBendingEnergy} 中的证明类似,只需
  设$\eta(x)=g(x)-f(x)$,并对
  $\int_{x_{1}}^{x_{N}}\lambda(t) D^{m}f(t) D^{m}\eta(t) \mathrm{d}t$做 $m$ 次分部积分,
  利用$f$的自然边界性质,$\eta$在$x_{i}$处为0,以及$D^{2m}f\equiv 0$,
  可以证明此积分式为0。
\end{proof}

现在,我们开始此问题的证明。

\begin{thm}
  \label{thm:FINAL2mSmoothingSplinesFitting}
  假设已知$\rho$的值,则
  存在唯一满足 (\ref{eq:FINAL2mSmoothingSplinesFittingEquiv})的$f\in  X_{\textnormal{BSSF}}$,
  且其满足自然边界条件 (\ref{eq:FINALnaturalOrder2m})。
\end{thm}
\begin{proof}
  假设存在$f\in X_{\textnormal{BSSF}}$为满足问题的解。我们对
  数据点$\{(x_{i},f(x_{i}))\}_{i=1}^{N}$在空间$X_{\textnormal{BSSF}}$下进行
  $2m$阶自然样条插值,得到样条$s(f;\mathbf{x})$。则$E(f)=E(s)$,
  且由引理~\ref{lem:FINALminimumBendingEnergyForBSpline},
  $F(D^{m}s)\le F(D^{m}f)$,等号成立当且仅当$f=s$,即$f$为自然样条。
  由于我们需要保证$f$最小化$\rho E(f)+F(D^{m}f)$,因此
  只能$f=s$。故$f$唯一,且是自然样条。

  接下来证$f$的存在性。
  由于$f$的求解与$\rho$有关,为方便说明,我们记问题的解为$f_{\rho}$,
  而后续提及的$f$可能是$ X_{\textnormal{BSSF}}$或其他空间的任意函数。
  同时,记$a:=x_{1},\ b:=x_{N},\ X:=X_{\textnormal{BSSF}}$。
  我们利用内积空间最佳逼近理论来
  寻找~(\ref{eq:FINAL2mSmoothingSplinesFittingEquiv})中的$f_{\rho}$。
  \cite{Boor2001CALCULATIONOT}

  定义线性映射
\begin{align*}
  \alpha:X\to Y&:=\mathbb{R}^{N}:\quad
                 f\mapsto f|_{\mathbf{x}}:=
                 [f(x_{1}),\cdots,f(x_{N})]^{T},\\
  \beta:X\to Z&:=L_{2}[a,b]:\quad
                f\mapsto D^{m}f,
\end{align*}
由于$Y$和$Z$均为 Hilbert 空间,由
推论 \ref{coro:FINALtwoHilbertSpacesProduct} 知,
$H:=Y\times Z$为 Hilbert 空间,且我们定义
$H$上的内积为
\begin{displaymath}
  \langle (\mathbf{f},g),(\mathbf{h},k)\rangle
  :=\rho\langle \mathbf{f},\mathbf{h}\rangle_{Y}
  +\langle g,k\rangle_{Z},
\end{displaymath}
其中
\begin{align*}
  \langle \mathbf{f},\mathbf{h}\rangle_{Y}
  &:=\sum_{j=1}^{N}w_{j}f_{j}h_{j},
    \quad \mathbf{f},\mathbf{h}\in Y,\\
  \langle g,k\rangle_{Z}&:=\int_{a}^{b}\lambda(t) g(t)k(t)\mathrm{d}t,
                          \quad g,k\in Z.
\end{align*}
注意到
\begin{align*}
  \rho E(f_{\rho})+F(D^{m}f_{\rho})
  &=\rho\sum_{j=1}^{N}w_{j}(y_{j}-f_{\rho}(x_{j}))^{2}
    +\int_{a}^{b}\lambda(t)|D^{m}f_{\rho}(t)|^{2}\mathrm{d}t\\
  &=\rho\|\mathbf{y}-\alpha(f_{\rho})\|_{Y}^{2}
    +\|\beta(f_{\rho})\|_{Z}^{2}\\
  &=\|(\mathbf{y},0)-(\alpha(f_{\rho}),\beta(f_{\rho}))\|^{2},
\end{align*}
为了使用定理 \ref{thm:FINALProjectionOp},
我们需要证明以下两个命题:
\begin{itemize}
\item $ X\to H:\ f\mapsto \left(\alpha(f),\beta(f)\right)$ 是一个嵌入
  (从而可以将 $X$视为$H$的向量子空间);
  \item $X$是闭的(从而可以将$X$视为$H$的闭向量子空间)。
\end{itemize}
对于第一个命题,其充要条件是
\begin{displaymath}
  \ker\alpha \cap \ker \beta=\{0\},
\end{displaymath}
并且由于
\begin{displaymath}
  \ker\alpha =\{f\in X:\ f|_{\mathbf{x}}=0\},\quad
  \ker \beta=\mathbb{P}_{m-1},
\end{displaymath}
故第一个命题成立当且仅当$N\ge m$。
在一般情形下,数据点的个数$N$会很大,故在今后假设$N\ge m$。
对于第二个命题,我们考虑$f\in X$在$a$处的带积分余项的$m$阶 Taylor 展开:
\begin{equation}
  \label{eq:FINALstandardRepresentationX}
  f=T_{a,m}f+R\beta(f),
\end{equation}
其中$T_{a,m}f$是$f$在$a$处的$m$阶 Taylor 多项式,且
\begin{displaymath}
  R:Z\to X:\
  g\mapsto \frac{1}{(m-1)!}
  \int_{a}^{b}(\cdot -s)^{m-1}_{+}g(s)\mathrm{d}s.
\end{displaymath}
从而$X=\mathbb{P}_{m-1}+R(Z)$为两个闭子空间的和,仍然是闭的。

因此$f_{\rho}$是从$X\subset H$到元素
$(\mathbf{y},0)\in H$的唯一最佳逼近,且
$(\mathbf{y},0)-(\alpha(f_{\rho}),\beta(f_{\rho}))$垂直于$X$,即
\begin{equation}
  \label{eq:FINALerrorPerpX}
  \forall f\in X,\quad
  \rho \langle \mathbf{y}-\alpha(f_{\rho}),\alpha(f) \rangle_{Y}
  -\langle \beta(f_{\rho}),\beta(f) \rangle_{Z}=0.
\end{equation}
由于$\ker\beta=\mathbb{P}_{m-1}$,~(\ref{eq:FINALerrorPerpX}) 意味着
\begin{equation}
  \label{eq:FINALerrorPerpPolym}
  \forall f\in \mathbb{P}_{m-1},\quad
  \langle \mathbf{y}-\alpha(f_{\rho}),\alpha(f) \rangle_{Y}=0.
\end{equation}
以及,结合~(\ref{eq:FINALstandardRepresentationX})和上式,
(\ref{eq:FINALerrorPerpX}) 意味着
\begin{equation}
  \label{eq:FINALerrorPerpZ}
  \forall g\in Z,\quad
  \rho\langle \mathbf{y}-\alpha(f_{\rho}),\alpha(Rg) \rangle_{Y}
  =\langle \beta(f_{\rho}),g \rangle_{Z}.
\end{equation}
反之,~(\ref{eq:FINALerrorPerpPolym}) 和~(\ref{eq:FINALerrorPerpZ})
 意味着~(\ref{eq:FINALerrorPerpX})。

由于~(\ref{eq:FINALerrorPerpZ}) 的左边作为$g$的函数是一个连续线性泛函,
故它可以表示为$\langle z,g \rangle_{Z}$,其中$z\in Z$。
具体地,
\begin{align*}
    \langle \mathbf{y}-\alpha(f_{\rho}),\alpha(Rg) \rangle_{Y}
  &=  \sum_{j=1}^{N}w_{j}(y_{j}-f_{\rho}(x_{j}))
    \frac{1}{(m-1)!}\int_{a}^{b}(x_{j}-s)_{+}^{m-1}g(s)\mathrm{d}s\\
  &=  \int_{a}^{b}\frac{1}{(m-1)!}
      \sum_{j=1}^{N}w_{j}(y_{j}-f_{\rho}(x_{j}))(x_{j}-s)_{+}^{m-1}
    g(s)\mathrm{d}s\\
  &=  \left\langle \frac{1}{\lambda (m-1)!}
    \sum_{j=1}^{N}w_{j}(y_{j}-f_{\rho}(x_{j}))(x_{j}-\cdot)_{+}^{m-1},g\right\rangle_{Z}.
\end{align*}
因此,~(\ref{eq:FINALerrorPerpZ}) 等价于
\begin{equation}
  \label{eq:FINALmthderfrho}
  \frac{\rho}{(m-1)!}\sum_{j=1}^{N}
  w_{j}(y_{j}-f_{\rho}(x_{j}))(x_{j}-\cdot)_{+}^{m-1}
  =\lambda D^{m}f_{\rho}.
\end{equation}
这表明$\lambda D^{m}f_{\rho}$是阶数为$m$的分段多项式,
其节点序列为$\mathbf{x}$,
这也说明了$f_{\rho}$为何是属于$X$的。
此外,不难发现$\lambda D^{m}f_{\rho}$在$x_{N}$及其右侧为0。

从另一角度看,由~(\ref{eq:FINALerrorPerpPolym}),
\begin{displaymath}
  \frac{\rho}{(m-1)!}\sum_{j=1}^{N}
  w_{j}(y_{j}-f_{\rho}(x_{j}))(x_{j}-\cdot)^{m-1}=0,
\end{displaymath}
因此,结合$(x_{j}-t)_{+}^{m-1}
=(x_{j}-t)^{m-1}-(-1)^{m-1}(t-x_{j})_{+}^{m-1}$,也有
\begin{equation}
  \label{eq:FINALmthderfrhoEquiv}
  \frac{\rho (-1)^{m}}{(m-1)!}\sum_{j=1}^{N}
  w_{j}(y_{j}-f_{\rho}(x_{j}))(\cdot-x_{j})_{+}^{m-1}
  =\lambda D^{m}f_{\rho},
\end{equation}
表明$\lambda D^{m}f_{\rho}$也在$x_{1}$及其左侧为0。

另外,由于$f_{\rho}$为自然样条,我们还有
\begin{displaymath}
  \forall j=0,1,\cdots,m-2,\quad
  D^{j}(\lambda D^{m}f_{\rho})(x_{1})=
  D^{j}(\lambda D^{m}f_{\rho})(x_{N})=0,
\end{displaymath}
这导致了$\lambda D^{m}f_{\rho}$的生成元
不会有那些通过重复节点定义的$m$阶 B 样条基函数。
具体地,对于$0<k<m$和节点组
$t_{0}=t_{1}=\cdots=t_{k}<t_{k+1}<\cdots<t_{m}$,
成立$D^{m-k-1}B_{1}(t_{0})\neq 0$.

因此,$\lambda D^{m}f_{\rho}$仅需在
$\{B_{2},\cdots,B_{N-m+1}\}$所张成的空间里,其中
$B_{j+1}$为对应于节点$\{x_{k}\}_{k=j}^{j+m}$的
B 样条基函数,
\begin{equation}
  \label{eq:FINALmthderfrhoLinearComp}
  \begin{aligned}
    B_{k+1}(t)
    =(x_{k+m}-x_{k})[x_{k},\cdots,x_{k+m}](\cdot - t)_{+}^{m-1}
    =:\sum_{j=k}^{k+m}(x_{j}-t)_{+}^{m-1}c_{j,k},
  \end{aligned}
\end{equation}
其中,由差商性质,
\begin{displaymath}
  c_{j,k}=
    \begin{cases}
      \frac{x_{k+m}-x_k}{\prod\{ (x_{j}-x_{i}):i\in\{k,\cdots,k+m\}
      \backslash\{j\} \}}, \quad &j=k,\cdots,k+m;\\
      0,\quad&\text{其他}.
    \end{cases}
\end{displaymath}
  
令$\mathbf{c}=[c_{1},\cdots,c_{N-m}]^{T}\in \mathbb{R}^{N-m}$,待定系数
\begin{equation}
  \label{eq:FINALlambdaOrdermfrhoAsBsplineSpan}
  \lambda D^{m}f_{\rho}=\sum_{k=1}^{N-m}c_{k}B_{k+1}.
\end{equation}
将~(\ref{eq:FINALmthderfrhoLinearComp})
代入~(\ref{eq:FINALlambdaOrdermfrhoAsBsplineSpan}),
并与~(\ref{eq:FINALmthderfrho}) 中$(x_{j}-t)_{+}^{m-1}$的系数做对比,得到
\begin{equation}
  \label{eq:FINALrelationOfCc}
  \rho W(\mathbf{y}-\alpha(f_{\rho}))=C\mathbf{c},
\end{equation}
其中
\begin{displaymath}
  W:=\text{diag}(\mathbf{w}),\quad
  C:=(m-1)!(c_{j,k})_{1\le j\le N,1\le k\le N-m}.
\end{displaymath}

现在我们证明~(\ref{eq:FINALrelationOfCc}) 与~(\ref{eq:FINALerrorPerpX}) 等价。
一方面,
由于~(\ref{eq:FINALrelationOfCc}) 与~(\ref{eq:FINALmthderfrhoLinearComp})
推出~(\ref{eq:FINALmthderfrho}),从而推出~(\ref{eq:FINALerrorPerpZ})。
另一方面,注意到,对任意$f\in X$,
\begin{equation}
  \label{eq:FINALPeanoKernelForDividedDifference}
  \begin{aligned}
     (C^{T}\alpha(f))_{j}
    &= (m-1)!\sum_{k=j}^{j+m}c_{k,j}f(x_{k})\\
    &= (m-1)!(x_{j+m}-x_{j})
       \frac{f(x_{k})}{\prod_{i=j;i\neq k}^{j+m}(x_{k}-x_{i})}\\
   &= (m-1)!(x_{j+m}-x_{j})\cdot [x_{j},\cdots,x_{j+m}]f\\
    &= (x_{j+m}-x_{j})\cdot [x_{j},\cdots,x_{j+m}]
      \int_{a}^{b}(t-s)_{+}^{m-1}D^{m}f(s)\mathrm{d}s\\
    &= \int_{a}^{b}(x_{j+m}-x_{j})[x_{j},\cdots,x_{j+m}](t-s)_{+}^{m-1}
      \cdot D^{m}f(s)\mathrm{d}s\\
    &= \int_{a}^{b}B_{j+1}(s)D^{m}f(s)\mathrm{d}s,
  \end{aligned}
\end{equation}
其中
第四步来自~(\ref{eq:FINALstandardRepresentationX}),
最后一步来自~(\ref{eq:FINALmthderfrhoLinearComp})。
因此,
\begin{align*}
  \forall f\in X,\quad
  \langle \mathbf{y}-\alpha(f_{\rho}),\alpha(f) \rangle_{Y}
  &=\alpha(f)^{T}W(\mathbf{y}-\alpha(f_{\rho}))\\
  &=\alpha(f)^{T}C \frac{\mathbf{c}}{\rho}
  =(C^{T}\alpha(f))^{T}\frac{\mathbf{c}}{\rho},
\end{align*}
而由~(\ref{eq:FINALPeanoKernelForDividedDifference}),
对$f\in \mathbb{P}_{m-1}$有$C^{T}\alpha(f)=0$。
从而~(\ref{eq:FINALrelationOfCc}) 意味着~(\ref{eq:FINALerrorPerpPolym})。
这表明,
(\ref{eq:FINALrelationOfCc}) 能推出
~(\ref{eq:FINALerrorPerpPolym}) 和~(\ref{eq:FINALerrorPerpZ}),
从而推出~(\ref{eq:FINALerrorPerpX})。
现在,我们仅需关注~(\ref{eq:FINALrelationOfCc})。

为此,从~(\ref{eq:FINALPeanoKernelForDividedDifference}) 与
(\ref{eq:FINALlambdaOrdermfrhoAsBsplineSpan}) 得,
\begin{equation}
  \label{eq:FINALrelationOfCtc}
  C^{T}\alpha(f_{\rho})=A\mathbf{c},
\end{equation}
其中
\begin{displaymath}
  A:= \left( \int_{a}^{b}\frac{1}{\lambda(t)}B_{j+1}(t)B_{k+1}(t)
    \mathrm{d}t
  \right)_{1\le j,k \le N-m}.
\end{displaymath}
将~(\ref{eq:FINALrelationOfCc}) 以
\begin{equation}
  \label{eq:FINALfrhoValueFromc}
  \alpha(f_{\rho})=\mathbf{y}
  -W^{-1}C\cdot \frac{\mathbf{c}}{\rho}
\end{equation}
的形式代入~(\ref{eq:FINALrelationOfCtc}),得到以下矩阵方程:
\begin{equation}
  \label{eq:FINALlineareqnOfc}
  C^{T}\mathbf{y}=(C^{T}W^{-1}C+\rho A)\mathbf{u},
\end{equation}
其中$\mathbf{u}:=\mathbf{c}/\rho$。
由于该方程的系数矩阵是对称正定的,因此$\mathbf{u}$存在唯一。
并且,由 (\ref{eq:FINALfrhoValueFromc}) 能够求出$\alpha(f_{\rho})$。
为了得到$f_{\rho}$,将得到的结果
$D^{m}f_{\rho}=(1/\lambda)\sum_{j=1}^{N-m}c_{j}B_{j+1}$积分$m$次,
得到$f:=D^{-m}(D^{m}f_{\rho})$,它与$f_{\rho}$仅相差某个
$q\in \mathbb{P}_{m-1}$。
可以确定这个$q$为唯一的$q\in \mathbb{P}_{m-1}$,使得
$\alpha(q+f)=\alpha(f_{\rho})$,即
$\alpha(q)=\alpha(f_{\rho})-\alpha(f)$。
\end{proof}

  求解方程 (\ref{eq:FINALlineareqnOfc}) 时,当$m$越来越大时,
  $C$和$A$的装配会极其困难。我们仅实现$m=1,2,3$的情形。
  为方便起见,我们把$A$和$C$的元素列于下方。
  \begin{align*}
    A_{j,k}&= \int_{a}^{b}\frac{1}{\lambda(t)}B_{j+1}(t)B_{k+1}(t)\mathrm{d}t,\quad
             1\le j,k\le N-m,\\
    C_{j,k}&=
    \begin{cases}
      \frac{(m-1)!(x_{k+m}-x_k)}{\prod\{ (x_{j}-x_{i}):i\in\{k,\cdots,k+m\}
      \backslash\{j\} \}}, \quad &j=k,\cdots,k+m;\\
      0,\quad&\text{其他},
    \end{cases}\quad
     1\le j\le N,\ 1\le k\le N-m.
  \end{align*}
  下面令$h_{i}:=x_{i+1}-x_{i}$,$\tilde{h}_{i}:=h_{i}/\lambda_{i}$。
  
  当$m=1$时,$B_{j+1}=\chi_{[x_{j},x_{j+1})}$,从而
  \begin{displaymath}
    \int_{x_{j}}^{x_{j+1}}B_{j+1}(t)B_{k+1}(t)\mathrm{d}t=\delta_{j,k}h_{j}.
  \end{displaymath}
 % 其中$\delta_{j,j}=1,\ \delta_{j,k}=0,\ \forall j\neq k$。
  故$A$是对角矩阵,其对角元为
  $\tilde{h}_{j},\ j=1,\cdots,N-1$。
  此外,直接计算得:$C_{k,k}=-1,C_{k+1,k}=1,\ \forall k$,
  其他$C$中的元素均为0。

  当$m=2$时,$B_{j+1}$是分片线性函数,在$x_{j}$上取值为1,
  在$x_{i},\ i\neq j$时取值为0。
  令$t=x_{j}+sh_{j}$,则
  \begin{displaymath}
    \int_{x_{j}}^{x_{j+1}}B_{j}(t)^{2}\mathrm{d}t
    =h_{j}\int_{0}^{1}s^{2}\mathrm{d}s=\frac{h_{j}}{3},
  \end{displaymath}
  而
  \begin{displaymath}
    \begin{aligned}
      \int_{x_{j}}^{x_{j+1}}B_{j-1}(t)B_{j}(t)\mathrm{d}t
      &=\int_{x_{j}}^{x_{j+1}}B_{j}(t)\mathrm{d}t
      -\int_{x_{j}}^{x_{j+1}}B_{j}(t)^{2}\mathrm{d}t\\
      &=\frac{h_{j}}{2}-\frac{h_{j}}{3}
      =\frac{h_{j}}{6}.
    \end{aligned} 
  \end{displaymath}
  因此,$A$是三对角矩阵,其一般行形式为
  \begin{displaymath}
    \frac{1}{6}
    \left(\tilde{h}_{j},
      2\left(\tilde{h}_{j}+\tilde{h}_{j+1}\right),\tilde{h}_{j+1}
    \right),
  % \left( \frac{h_{j}}{\lambda_{j}},
  %   2 \left( \frac{h_{j}}{\lambda_{j}}+
  %     \frac{h_{j+1}}{\lambda_{j+1}}\right),
  %   \frac{h_{j+1}}{\lambda_{j+1}}
  % \right),
  \qquad j=1,\cdots, N-2.
  \end{displaymath}
  此外,直接计算得:
  \begin{align*}
    \forall k,\qquad
    C_{k,k}=\frac{1}{h_{k}},\quad
    C_{k+1,k}=-\frac{1}{h_{k}}-\frac{1}{h_{k+1}},\quad
    C_{k+2,k}=\frac{1}{h_{k+1}},\quad
  \end{align*}
  其他$C$中的元素均为$0$。

  当$m=3$时,直接计算得:
  \begin{align*}
    \forall k,\qquad
    C_{k,k}
    =-\frac{2}{h_{k}(h_{k}+h_{k+1})},&\quad
    C_{k+1,k}
    =\frac{2(h_{k}+h_{k+1}+h_{k+2})}{h_{k}h_{k+1}(h_{k+1}+h_{k+2})},\\
    C_{k+2,k}
    =-\frac{2(h_{k}+h_{k+1}+h_{k+2})}{(h_{k}+h_{k+1})h_{k+1}h_{k+2}},&\quad
    C_{k+3,k}=\frac{2}{(h_{k+1}+h_{k+2})h_{k+2}},
  \end{align*}
  其他$C$中的元素均为$0$。接下来计算$A$。
  直接计算显然不是好方法,因为$B_{j}$在每段区间上是$\mathbb{P}_{2}$的,
  这需要计算5个4次多项式的积分。
  我们采用三点 Gauss-Legendre 数值积分,它有5阶代数精度:
  \begin{displaymath}
    \int_{x_{i}}^{x_{i+1}}g(t)\mathrm{d}t
    =h_{i}\left(
    \frac{5}{18}g\left( l_{i} \right)
    +\frac{4}{9}g(m_{i})
    +\frac{5}{18}g\left( r_{i} \right)
    \right),
  \end{displaymath}
  其中
  \begin{align*}
    m_{i}=\frac{(x_{i}+x_{i+1})}{2},\quad
    l_{i}=m_{i}-\sqrt{\frac{3}{5}}\frac{h_{i}}{2},\quad
    r_{i}=m_{i}+\sqrt{\frac{3}{5}}\frac{h_{i}}{2}.
  \end{align*}
  因此
  \begin{align*}
    A_{j,k}
    &=\sum_{i=1}^{N-1}\int_{x_{i}}^{x_{i+1}}\frac{1}{\lambda(t)}B_{j+1}(t)B_{k+1}(t)\\
    &=\sum_{i=1}^{N-1}\tilde{h}_{i}\left(
    \frac{5}{18}B_{j+1}(l_{i})B_{k+1}(l_{i})
   +\frac{4}{9}B_{j+1}(m_{i})B_{k+1}(m_{i})
     +\frac{5}{18}B_{j+1}(r_{i})B_{k+1}(r_{i})
    \right).
  \end{align*}
  具体地,
  结合$B_{k+1}$仅在$(x_{k},x_{k+3})$上不为0,得:对$\forall k$,
  \begin{align*}
    &\ A_{k,k}
    =\sum_{i=k}^{k+2}\tilde{h}_{i}\left(
      \frac{5}{18}B_{k+1}^{2}(l_{i}) 
      +\frac{4}{9}B_{k+1}^{2}(m_{i})
      +\frac{5}{18}B_{k+1}^{2}(r_{i})
      \right),\\
    &\ A_{k-1,k}
    =\sum_{i=k}^{k+1}\tilde{h}_{i}\left(
    \frac{5}{18}B_{k}(l_{i})B_{k+1}(l_{i})
     +\frac{4}{9}B_{k}(m_{i})B_{k+1}(m_{i})
     +\frac{5}{18}B_{k}(r_{i})B_{k+1}(r_{i})\right),\\
    &\
    A_{k-2,k}
    =\tilde{h}_{k}\left(
    \frac{5}{18}B_{k-1}(l_{k})B_{k+1}(l_{k}) 
    +\frac{4}{9}B_{k}(m_{k})B_{k+1}(m_{k})
    +\frac{5}{18}B_{k}(r_{k})B_{k+1}(r_{k})
    \right),
  \end{align*}
  其他$A$中的元素均为0。在这些表达式中,
  诸如$B_{k}(m_{i})$之类的计算在配置矩阵的计算中有体现,可以直接计算。

  接下来,我们来讲述计算$\rho$的方法。
  记$E_{\rho}:=\|\mathbf{y}-\alpha(f_{\rho})\|_{Y}^{2}$。则$\rho$满足
  以下方程:
  \begin{equation}
    \label{eq:FINALrhoEqn}
    g(\rho):=
    \frac{1}{E_{\rho}^{\frac{1}{2}}}
    -\frac{1}{S^{\frac{1}{2}}}=0.
  \end{equation}
  选取该方程进行迭代求解,可以使求解效率更高,更快收敛,
  具体细节可见\cite{Boor2001CALCULATIONOT,SmoothingBySplineFunctions}。
  由~(\ref{eq:FINALfrhoValueFromc})得
\begin{displaymath}
  E_{\rho}=\mathbf{u}^{T}C^{T}W^{-1}C\mathbf{u}.
\end{displaymath}
记求导算子$D:=\mathrm{d}/\mathrm{d}\rho$,则
$DE_{\rho}=2\mathbf{u}^{T}C^{T}W^{-1}CD\mathbf{u}$。
将$D$作用于~(\ref{eq:FINALlineareqnOfc})并整理,可知
$D\mathbf{u}$通过线性系统
\begin{displaymath}
  -A\mathbf{u}=(C^{T}W^{-1}C+\rho A)D\mathbf{u}
\end{displaymath}
唯一确定。
特别地,对$\rho=0$,这表示$DE_{\rho}=-2\mathbf{u}^{T}A\mathbf{u}$,
其中$\mathbf{u}=(C^{T}W^{-1}C)^{-1}C^{T}\mathbf{y}$,
这为在初始步骤$\rho_{0}=0$处对~(\ref{eq:FINALrhoEqn})
应用牛顿法提供了所需的斜率:
\begin{displaymath}
  g'(0)=-\frac{1}{2}E_{0}^{-\frac{3}{2}}DE_{0}
  =E_{0}^{-\frac{3}{2}}\mathbf{u}^{T}A\mathbf{u}.
\end{displaymath}
其迭代格式为:
\begin{displaymath}
%  \label{eq:rhoNewtonIter}
  \rho_{1}=\rho_{0}-\frac{g(\rho_{0})}{g'(\rho_{0})}
  =-\frac{g(0)E_{0}^{\frac{3}{2}}}{\mathbf{u}^{T}A\mathbf{u}}.
\end{displaymath}
对于后续步骤,为避免计算$DE_{\rho}$
(这需要解一个线性系统),我们使用割线法。其迭代格式为:
\begin{align*}
  % \label{eq:rhoSecantIter}
  \forall n\ge 1, \quad
  \rho_{n+1}
    =\rho_{n}
      -g(\rho_{n})\frac{\rho_{n}-\rho_{n-1}}{g(\rho_{n})-g(\rho_{n-1})}
    =\rho_{n}
     +\frac{\rho_{n}-\rho_{n-1}}{\frac{g(\rho_{n-1})}{g(\rho_{n})}-1}.
\end{align*}
总结$\rho_{n}$的迭代格式为:
\begin{equation}
  \label{eq:FINALrhoIter}
  \begin{aligned}
    \rho_{0}=0,\quad
    \rho_{1}
      =-\frac{g(0)E_{0}^{\frac{3}{2}}}{\mathbf{u}^{T}A\mathbf{u}},\quad
    \rho_{n+1}
    =\rho_{n}+\frac{\rho_{n}-\rho_{n-1}}{\frac{g(\rho_{n-1})}{g(\rho_{n})}-1},
      \quad n\ge 1.
  \end{aligned}
\end{equation}
迭代的终止条件设定为相对误差1\%以内,即
\begin{equation}
  \label{eq:FINALrhoIterEnd}
  100|E_{\rho_{n}}-S|<S.
\end{equation}

综上,我们给出解决$2m$阶平滑样条拟合的算法 \ref{alg:FINALBSSF}。
\begin{algorithm}[htb]
\caption{$2m$阶平滑样条拟合:$f=\textnormal{spaps}(\mathbf{x}, \mathbf{y}, \mathbf{w}, \lambda, m,S)$}
\label{alg:FINALBSSF}
\begin{algorithmic}[1] 
  \renewcommand{\algorithmicrequire}{ \textbf{Input:}} %Use Input in the format of Algorithm
  \REQUIRE  %算法的输入参数:Input
  定义~\ref{def:FINAL2mSmoothingSplinesFitting} 中的$\mathbf{x}, \mathbf{y}, \mathbf{w}, \lambda, m,S$
    
      \renewcommand{\algorithmicrequire}{ \textbf{Preconditions:}} %Use Input in the format of Algorithm
  %\REQUIRE  %算法的输入参数:Preconditions
      
    \renewcommand{\algorithmicensure}{ \textbf{Output:}} %UseOutput in the format of Algorithm
  \ENSURE  %算法的输出:Output
  定理 \ref{thm:FINAL2mSmoothingSplinesFittingEquiv} 中的$f$和$\rho$
\renewcommand{\algorithmicensure}{ \textbf{Postconditions:}} %UseOutput in the format of Algorithm
\ENSURE  %算法的输出:Postconditions
    $f$为 B 格式,满足自然边界条件
    \STATE $A\leftarrow (\int_{a}^{b}\frac{1}{\lambda(t)}B_{j+1}(t)B_{k+1}(t)\mathrm{d}t)_{1\le j,k\le N-m}$;  
             $C\leftarrow (m-1)!(c_{j,k})_{1\le j\le N,1\le k\le N-m}$;
             $W\leftarrow \text{diag}(\mathbf{w})$
    \STATE $\mathbf{u}\leftarrow(C^{T}W^{-1}C)^{-1}C^{T}\mathbf{y}$
    \STATE $E\leftarrow \mathbf{u}^{T}C^{T}W^{-1}C\mathbf{u}$;
     $g_{0}\leftarrow E^{-\frac{1}{2}}-S^{-\frac{1}{2}}$;
     $\rho\leftarrow -g_{0}E^{\frac{3}{2}}/\mathbf{u}^{T}A\mathbf{u}$; $\Delta \rho\leftarrow\rho$
     \hfill // $\rho$的第一步迭代
    \WHILE{$\Delta\rho > 0$}
    \STATE $\mathbf{u}\leftarrow(C^{T}W^{-1}C+\rho A)^{-1}C^{T}\mathbf{y}$
    \STATE $E\leftarrow \mathbf{u}^{T}C^{T}W^{-1}C\mathbf{u}$
    \IF{$100|E-S| < S$}
        \STATE \textbf{break}
    \ENDIF
    \STATE $g_{\rho}\leftarrow {E}^{-\frac{1}{2}} - S^{-\frac{1}{2}}$;
     $\Delta\rho\leftarrow \Delta\rho / (g_{0}/g_{\rho} - 1)$;
     $g_{0} \leftarrow g_{\rho}$;
     $\rho \leftarrow\rho + \Delta\rho$
     \ENDWHILE
     \STATE $\alpha(f_{\rho})\leftarrow \mathbf{y}-W^{-1}C\mathbf{u}$
     \STATE $\mathbf{c}\leftarrow \rho \mathbf{u}$;
     $f\leftarrow D^{-m}((1/\lambda)\Sigma_{i=1}^{N-m}c_{j}B_{j+1})$
     \STATE $\alpha(q)\leftarrow \alpha(f_{\rho})-\alpha(f)$
     \STATE $q\leftarrow \text{PolyFitter}(\mathbf{x},\alpha(q))$
     \hfill // 函数PolyFitter返回拟合$N$个点所得的$m$阶多项式$q$
     \STATE $f\leftarrow f+q$
\end{algorithmic}
\end{algorithm}

在后续的数值测试(见第 \ref{sec:FINALTest} 节 )中,
我们发现所求出的 B 样条的基函数对应系数的精度下降地很快。
实际上,当区间$[a,b]$固定时,随着$N$的增大,步长$h_{i}$不断地减小,
而当$m=2$时,
矩阵$C$中的元素均为$O(h^{-1})=O(N)$的,这导致方程系数矩阵
$C^{T}W^{-1}C+\rho A$的条件数平方增长,求解不稳定。
