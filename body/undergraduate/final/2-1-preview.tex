\cleardoublepage

\section{预备知识}
\label{sec:FINALbuildingknowledge}

\subsection{样条函数}
\label{subsec:FINALsplinespace}
\begin{defn}[样条函数空间]
  \label{def:FINALsplinespace}
  给定非负整数$n$,$k$,以及一个严格递增序列
  $\mathbf{x}=\{x_{i}\}_{i=1}^{N}$,称
  \begin{equation}
    \label{eq:FINALsplinespace}
    \mathbb{S}_{n}^{k}(\mathbf{x})
    =\{s\in C^{k}[a,b]:\
    \forall i\in [1,N-1],\
    s|_{[x_{i},x_{i+1}]}\in \mathbb{P}_{n}\}
  \end{equation}
  为次数为$n$(阶数为$n+1$),平滑度为$k$的样条函数空间。
  称$\mathbf{x}$为样条的节点(knots)。
\end{defn}

最常见的样条函数的格式有两种: pp 格式(ppForm)和
B 格式(BForm)。
pp 样条存储一个$(N-1)$段的分段$n$次多项式,
而 B 样条存储一组 B 样条基函数及其所对应的系数。
一般地,我们考虑平滑度比次数少1的样条空间。下面我们
来介绍一下这种 B 样条空间所对应的基函数。

\begin{defn}[ B 样条基函数]
  \label{def:FINALbasicBspline}
  我们定义递增节点组$\{t_{j+i}\}_{i=0}^{k}$上的$k$阶 \textnormal{B} 样条基函数
  $B_{j+1}(x)$为
  \begin{equation}
  \label{eq:FINALbasicBSpline}
  B_{j+1}(x)=
  (t_{j+k}-t_{j})\cdot [t_{j},t_{j+1},\cdots,t_{j+k}](t-x)_{+}^{k-1},
  \end{equation}
  其中
  $[t_{j},t_{j+1},\cdots,t_{j+k}]g$为
  $g$关于$\{t_{j+i}\}_{i=0}^{k}$的$k$阶差商($k$th divided difference),
  可以递归地定义为
  \begin{equation}
    \label{eq:FINALdividedDifference}
    [t_{j},t_{j+1},\cdots,t_{j+k}]g
    =\frac{[t_{j+1},t_{j+2},\cdots,t_{j+k}]g-
      [t_{j},t_{j+1},\cdots,t_{j+k-1}]g}{t_{j+k}-t_{j}};
  \end{equation}
  $t_{+}^{m}$为截断幂函数(truncated power function),定义为
  \begin{equation}
    \label{eq:FINALtruncatedPowerFunc}
    t_{+}^{m}=
    \left\lbrace
    \begin{aligned}
      t^{m},\ t&\ge 0,\\
      0,\ t&<0.
    \end{aligned}
    \right.
  \end{equation}
\end{defn}

根据广义差商(generalized divided difference)的知识,
$\{t_{j+i}\}_{i=0}^{k}$中是可以存在重复节点的。也就是说,在有重复节点
且$t_{j}<t_{j+k}$的情况下,(\ref{eq:FINALbasicBSpline}) 仍是良定义的。

一般地,假定$k$阶 B 样条空间
由$B_{2},B_{3},\cdots,B_{n+1}$这$n$个 B 样条基函数张成,
则需要用到的递增节点组为$\mathbf{t}:=\{t_{i}\}_{i=1}^{n+k}$。
这个时候,我们一般认为
\begin{displaymath}
  t_{1}=t_{2}=\cdots=t_{k}<t_{k+1}<\cdots<t_{n}<
  t_{n+1}=t_{n+2}=\cdots=t_{n+k},
\end{displaymath}
并视
\begin{displaymath}
  x_{i}=t_{i+k-1},\quad i=1,2,\cdots,n-k+2.
\end{displaymath}
那么,这个样条空间的节点被确定为$\{x_{i}\}_{i=1}^{n-k+2}$。
有时,我们也不考虑节点序列的严格递增性,而简单地将 B 样条空间
记为$\mathbb{S}_{k-1}^{k-2}(\mathbf{t})$。
不难验证,$B_{2},B_{3},\cdots,B_{n+1}\in \mathbb{S}_{k-1}^{k-2}(\mathbf{t})$。

当使用 B 样条对数据点进行插值或拟合时,样条配置矩阵
(spline collocation matrix)的概念是非常重要的。
\begin{defn}[样条配置矩阵]
  \label{def:FINALSplineCollocationMatrix}
  假设 \(\mathbf{t} = \left\{t_i\right\}_{i=1}^{n+k}\) 是一个递增节点序列,
  并且 \(\left\{B_i\right\}_{i=2}^{n+1}\) 是对应的$k$阶 \textnormal{B} 样条序列。
  如果给定$N\ge n$,以及
  严格递增序列 \(\mathbf{x} = \left\{x_i\right\}_{i=1}^N\),
  那么对于给定的值 \(\mathbf{y} = \left\{y_i\right\}_{i=1}^N\),
  样条 \(f = \sum_{j=1}^n a_j B_{j+1}\) 满足
  \(f(x_i) = y_i, i = 1,\dots,N\) 当且仅当
  \begin{equation}
    \label{eq:FINALSplineCollocationMatrix}
    \sum_{j=1}^n a_j B_{j+1}(x_i) = y_i, \quad i = 1,2,\dots,N.
  \end{equation}
  这是一个关于未知数 \(\mathbf{a} = \left\{a_i\right\}_{i=1}^n\) 的
  \(N\) 个方程的线性系统,其系数矩阵 \((B_{j+1}(x_i))\)$_{N\times n}$ 称为样条配置矩阵。
\end{defn}

当$N>n$时,方程组 \eqref{eq:FINALSplineCollocationMatrix} 是超定的,
在最小二乘的意义下,其存在唯一解当且仅当其样条配置矩阵(记为$A$)是列满秩的,
当且仅当存在$1\le j_{1}<j_{2}<\cdots<j_{n}\le N$,使得$A$的
第$j_{1},j_{2},\cdots,j_{n}$行组成的子矩阵$\tilde{A}$是可逆的。

当$N=n$时,Schoenberg-Whitney 定理指出了样条配置矩阵可逆的
等价条件\cite{GuideToSplines}: 

\begin{thm}[Schoenberg-Whitney]
  \label{thm:FINALSchoenberg-Whitney}
  样条配置矩阵$(B_{j+1}(x_i))_{n\times n}$是可逆的,当且仅当
  Schoenberg-Whitney 条件成立,即数据点序列
  $\mathbf{x}$和节点序列$\mathbf{t}$满足
  \begin{equation}
    \label{eq:FINALSchoenberg-Whitney}
    t_i < x_i < t_{i+k}, \quad j = 1,\dots,n,
  \end{equation}
  其中在$t_{1}$和$t_{n+k}$处可以取等号。
\end{thm}

因此,我们汇总成如下定理:
\begin{thm}
  \label{thm:FINALSWSplinesApproximation}
  假设 \(\mathbf{t} = \left\{t_i\right\}_{i=1}^{n+k}\) 是一个递增节点序列,
  并且 \(\left\{B_i\right\}_{i=2}^{n+1}\) 是对应的$k$阶 \textnormal{B} 样条序列。
  如果给定$N\ge n$,以及
  严格递增序列 \(\mathbf{x} = \left\{x_i\right\}_{i=1}^N\),
  那么样条配置矩阵$(B_{j+1}(x_i))_{N\times n}$是列满秩的,当且仅当
  Schoenberg-Whitney 条件成立,即存在$\mathbf{x}$的子序列
  $\{x_{j_{i}}\}_{i=1}^{n}$,其和节点序列$\mathbf{t}$满足
  \begin{equation}
    \label{eq:FINALSWSplinesApproximation}
    t_i < x_{j_{i}} < t_{i+k}, \quad i = 1,\dots,n,
  \end{equation}
  其中在$t_{1}$和$t_{n+k}$处可以取等号。
\end{thm}


\subsection{最佳逼近和内积空间}
\label{sec:FINALBestApproximationAndLeastSquares}

\begin{defn}[最佳逼近,best approximation]
  \label{def:FINALBestApproximation}
  给定一个配备范数$\|\cdot\|$的函数线性空间$Y$
  及其子空间$X \subseteq Y$。称函数$\hat{\varphi} \in X$是
  从$X$中得到的对$f\in Y$的最佳逼近,如果
  \begin{equation}
    \label{eq:FINALBestApproximation}
    \hat{\varphi} = \arg\min_{\varphi\in X}\|f-\varphi\|,   
  \end{equation}
  也即
  \begin{equation}
    \label{eq:FINALBestApproximationEquiv}
    \forall \varphi \in X, \quad
    \|f - \hat{\varphi}\| \leq \|f - \varphi\|.
  \end{equation}
\end{defn}

\begin{defn}[离散内积]
  \label{def:FINALDiscreteInnerProduct}
  对于有限个互不相同的数据点$x_{1},x_{2},\cdots,x_{N}$,
  定义$\mathbb{R}$上的离散内积
  \begin{equation}
    \label{eq:FINALDiscreteInnerProduct}
    \langle u(x),v(x)\rangle
    =\sum_{i=1}^{N}w_{i}u(x_{i})v(x_{i}),
  \end{equation}
  其中函数$u,v:\mathbb{R}\to \mathbb{R}$,$w_{i}>0$为$x_{i}$处的权重。
  同时定义该内积诱导的离散范数$\|\cdot\|$:
  \begin{equation}
    \label{eq:FINALDiscreteNorm}
    \|u\|=\langle u,u\rangle^{\frac{1}{2}}=
    \left(\sum_{i=1}^{N}w_{i}u(x_{i})^{2}\right)^{\frac{1}{2}}.
  \end{equation}
\end{defn}

  为了使$\langle\cdot,\cdot\rangle$成为一个内积,
  还需要合理定义内积空间里的元素。合适的元素是函数类,其等价关系为:
  $u\sim \tilde{u} \Leftrightarrow
  u(x_{i})=\tilde{u}(x_{i}),\ i=1,2,\cdots,N$。
  当然我们也可以限制空间$X$,使此离散内积确实为一个内积,
  离散范数确实为一个范数。
  此情形在引理 \ref{lem:FINALSWconditionForDLSspline} 中有具体体现。

\subsection{Hilbert 空间理论}
\label{subsec:FINALHilbertspace}
  
  接下来,我们给出 Hilbert 空间的定义,其在
  \ref{sec:FINAL2mSmoothingSplinesFitting} 节中发挥作用。

  \begin{defn}[Hilbert 空间]
    \label{def:FINALHilbertSpace}
    设$H$是一内积空间,$\|\cdot\|= \langle \cdot,\cdot\rangle^{\frac{1}{2}}$
    为内积诱导的范数。若该范数是完备的,则称$H$是 Hilbert 空间。
  \end{defn}

  由 Hilbert 空间的定义,立马得到如下推论。
  \begin{coro}
    \label{coro:FINALtwoHilbertSpacesProduct}
    若$Y$, $Z$均为 Hilbert 空间,则$H:=Y\times Z$也是 Hilbert 空间。
  \end{coro}
  \begin{proof}
    记$Y$和$Z$上的内积分别为$\langle\cdot,\cdot\rangle_{Y}$,
    $\langle\cdot,\cdot\rangle_{Z}$。定义$H$上的内积为
    \begin{displaymath}
      \langle (f,g),(h,k)\rangle_{H}
      :=\langle f,h\rangle_{Y}
      +\langle g,k\rangle_{Z},
      \quad
      f,h\in Y,\ g,k\in Z.
    \end{displaymath}
    可以验证,该内积诱导出的范数也是完备的。
  \end{proof}

  \begin{thm}
    \label{thm:FINALProjectionOp}
    设$H$是 Hilbert 空间,$X$ 是 $H$的闭向量子空间,那么
    对任意$x\in H$,存在唯一的$x_{0}\in X$使得
    \begin{equation}
      \label{eq:FINALProjectionOp}
      \|x-x_{0}\|=d(x,X)
      :=\inf_{y\in X}\|x-y\|.
    \end{equation}
    此时有$x-x_{0}\perp X$,
    即$\forall y\in X,\ \langle x-x_{0},y\rangle=0$。
  \end{thm}

  
