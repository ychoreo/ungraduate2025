\section{问题提出的背景}
数值计算历史已发展久远。近年来,
数值计算方法被大范围地使用在解决理论数学问题,
甚至一些生活实际应用问题上。传统的数值计算方法只注重
数学理论的推导,缺乏了针对具体计算实现的训练。
因此,熟练地使用数值计算软件来编程得到
具体数值结果迫在眉睫。在当今的通用型科学计算软件中,
最常用的是 Matlab ,这类软件在工程和科学计算中有非常重要的应用。
长期以来,这类软件的市场由国外公司垄断。
通用型科学计算软件利用科学原理,
将各学科的问题通过计算机模拟或用其他形式展现,
从而辅助解决更多、更复杂的实际问题,
助力科研人员完成研究与工程设计工作。
一旦软件被全面停用,整个科研与工程设计进程将受到严重阻碍。
因此,突破“卡脖子”技术限制,研发出一款国产的科学计算软件,
一时间成为众多科研人员日夜奋斗的目标。
目前,我国科学家正在研发一款国产的、
具有自主知识产权的通用型科学计算软件——北太天元数值计算通用软件
(以下简称北太天元),
其目标是可以完全替代 Matlab 的功能。\cite{baltamChinese}

曲线拟合作为一种数学应用的基础方法,
被广泛应用在科研、工程、产品设计研发、数据分析等各个领域。
北太天元曲线拟合工具箱对标 Matlab 的曲线拟合工具箱,
近期目标是实现 Matlab 曲线拟合工具箱的完整功能函数。
因此,本研究的主要工作是,实现部分曲线拟合工具箱中的函数。

\section{本研究的意义和目的}

\par 曲线拟合在科学研究、工程应用和数据分析中具有重要地位。
其主要目的是通过已有数据点构建数学模型,
从而预测未知数据或理解数据之间的关系。
曲线拟合在以下领域尤为重要:
\begin{itemize}
\item 科学研究:用于分析实验数据,帮助科学家识别趋势和模式。
\item 工程设计:在机械、电子等领域,用于优化设计参数和提高系统性能。
\item 经济金融:用于市场趋势预测和风险管理。
\item 医学:帮助分析生物医学数据,支持临床决策。
\item 计算机图形学:用于生成平滑曲线和曲面,提高图形渲染质量。
\end{itemize}
特别地,使用样条函数进行曲线拟合,相比其他拟合方法具有一定的优势:
\begin{itemize}
\item 灵活性高:
  样条函数在不同区间上的多项式不同,能灵活地适应数据的局部特征。
\item 平滑性:
  样条函数通常是连续且平滑的,例如三次样条一般能够保证一阶和二阶导数的连续性。
\item 避免过拟合:
  相比高次多项式拟合,样条函数通过分段拟合减少了过拟合的风险。
\item 计算效率:
  用样条函数进行拟合,通常只需要较低阶的多项式,因此计算效率较高。
\item 局部控制:
  在曲线拟合中,修改样条函数的一个控制点,只会较大地影响相邻区间的形状,
  而对整个曲线的影响较小。
\end{itemize}
这些优势使得样条函数在处理复杂和不规则数据时非常有效。
