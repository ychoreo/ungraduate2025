\section{研究计划进度安排及预期目标}

\subsection{进度安排}
1月17日前,掌握用 B 样条解决离散最小二乘样条逼近问题以及
用 pp 样条解决三次平滑样条拟合问题的理论知识。

1月18日-2月20日,使用 C++ 语言编程出上述两个问题的代码。

2月21日-3月23日,掌握用 B 样条解决平滑样条拟合的理论知识,
完成中期检查报告。

3月24日-4月10日,使用 C++ 语言编程出上述问题的代码,
并将写过的代码与北太天元框架进行接合。

4月11日-4月27日,完成毕业论文初稿。

4月28日-5月25日,根据专家意见进行毕业论文修改,
并准备毕业论文答辩,提交毕业论文定稿。
\subsection{预期目标}
保证现有的代码中,每个函数的基本功能是正确的。
在预期时间内,掌握使用 B 样条实现$2m$阶平滑样条拟合的理论和编程方法。
在中后期,将自己编写的代码与北太天元接口对接后,进行充分的编程测试,
力求在北太天元测得的各指标与 MATLAB 中的几乎相同甚至更好。
希望在未来能够理解 MATLAB 曲线拟合工具箱中的更多函数的理论和编程方法。

