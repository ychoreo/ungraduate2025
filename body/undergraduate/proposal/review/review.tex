\cleardoublepage
\newrefsection

\chapter{文献综述}

\section{背景介绍}
\par 我们的研究对象为曲线拟合。
所谓曲线拟合,通俗地说,就是给定一组平面上的数据点,
如何找到一条曲线,使得该曲线能较好地拟合这些点。
然而,“较好地拟合”应当是什么样的拟合方式呢?
事实上,拟合的方式有很多种,它们是根据特定的拟合目标而制定出来的。
一般地,有以下几种比较经典的问题。
\begin{defn}[离散最小二乘(DLS)函数逼近,
  discrete least squares functions approximation]
  \label{def:DiscreteLeastSquaresFunctionsApproximation}
  若给定一组严格递增的数据点$\mathbf{x}=\{x_{i}\}_{i=1}^{N}$
  和对应的权重$\mathbf{w}=\{w_{i}\}_{i=1}^{N}$,
  其中$w_{1},w_{2},\cdots,w_{N}> 0$。则对于
  一组给定的值$\mathbf{y}=\{y_{i}\}_{i=1}^{N}$和一个给定的
  函数线性空间$X$,如果函数$f\in X$满足
  \begin{equation}
    \label{eq:DiscreteLeastSquaresFunctionsApproximation}
    f = \arg\min_{g\in X}\sum_{i=1}^{N}w_{i}(y_{i}-g(x_{i}))^{2},
    % \forall g\in X, \quad
    % \sum_{i=1}^{N}w_{i}(y_{i}-f(x_{i}))^{2}\le
    % \sum_{i=1}^{N}w_{i}(y_{i}-g(x_{i}))^{2},
  \end{equation}
  则称$f$为$\{(x_{i},y_{i})\}_{i=1}^{N}$在$X$下的
  离散最小二乘函数逼近。\cite{GuideToSplines}
\end{defn}

\begin{defn}[三次平滑函数拟合, cubic smoothing functions fitting]
  \label{def:cubicSmoothingFunctionsFitting}
  给定一组严格递增的数据点$\mathbf{x}=\{x_{i}\}_{i=1}^{N}$,一个
  函数线性空间$X$和一个平滑参数(smoothing parameter)$p\in [0,1]$。
  则对于一组给定的值$\mathbf{y}=\{y_{i}\}_{i=1}^{N}$,
  称函数$f\in X$是$\{(x_{i},y_{i})\}_{i=1}^{N}$在$X$下的
  三次平滑函数拟合\cite{GuideToSplines},如果
  \begin{equation}
    \label{eq:cubicSmoothingSpline}
    % \forall g\in X,\quad
    % pE(f)+(1-p)F(D^{2}f)\le
    f= \arg\min_{g\in X}\left( pE(g)+(1-p)F(D^{2}g)\right),
  \end{equation}
  其中
  \begin{align}
    \label{eq:errorMeasure}
    E(g)&:=\sum_{i=1}^{N}w_{i}(y_{i}-g(x_{i}))^{2},\\
    \label{eq:roughnessMeasure}
    F(D^{m}g)&:=\int_{x_{1}}^{x_{N}}
               \lambda(t)|D^{m}g(t)|^{2}\mathrm{d}t
  \end{align}
  分别称为$g$的误差度量(error measure)和
  粗糙度度量(roughness measure),
  $w_{i}>0$称为$x_{i}$处的权重误差(error weight),
  $\lambda(t)$称为粗糙度权重(roughness measure),
  是在$[x_{1},x_{N}]$上关于$\mathbf{x}$的分片常值函数,
  即
  \begin{equation}
    \label{eq:lambdat}
    \lambda(t)\equiv \lambda_{i}\ge 0,\
  t\in [x_{i},x_{i+1}],\quad i=1,2,\cdots,N-1.
  \end{equation}
\end{defn}

\begin{defn}[平滑函数拟合, smoothing functions fitting]
  \label{def:smoothingFunctionsFitting}
  沿用定义~\ref{def:cubicSmoothingFunctionsFitting} 中的记号,
  并给定正整数$m$和容忍度(tolerance)$S\ge 0$。
  如果存在函数$f\in X$,使得$E(f)\le S$,且
  \begin{equation}
    \label{eq:smoothingFunctionsFitting}
    f = \arg \min_{g\in X\text{ with } E(g)\le S} F(D^{m}g),
  \end{equation}
  则称$f$是$\{(x_{i},y_{i})\}_{i=1}^{N}$在$X$下误差度量不超过$S$的
  $2m$阶平滑函数拟合。\cite{SmoothingBySplineFunctions}
\end{defn}

自然地,我们想确定各定义中的函数空间。而样条函数拟合
相比于其他函数有一些优势,例如,
样条能通过局部节点划分实现灵活建模,规避全局
高阶多项式的 Runge 现象\cite{GuideToSplines,hastie2009elements};
样条一般是连续且平滑的,更易于做数据拟合\cite{green1993nonparametric};
利用样条计算的效率更高,
得到的结果更为稳定\cite{GuideToSplines,hastie2009elements}。
由于样条函数的良好性质,对于各定义中的问题,我们试图
在样条空间
$$\mathbb{S}_{n}^{n-1}(\mathbf{x})
=\{g\in C^{n-1}([x_{1},x_{N}]):
  \ g|_{[x_{i},x_{i+1}]}\in \mathbb{P}_{n},\ i=1,2,\cdots,N-1\}$$
或与其近似的空间中寻找一个样条函数,
这个函数能够解答所定义的问题。

DLS 样条逼近
(对应定义~\ref{def:DiscreteLeastSquaresFunctionsApproximation})
的提出来源于直接的 DLS 函数逼近。在此基础上也延伸了一些研究方向:
Baramidze 和 Lai \cite{BARAMIDZE20111091}
研究了稳定局部基空间中 DLS 球面样条的收敛性,
揭示了误差限对三角网格尺寸的依赖关系;
Cromme、Kunath 和 Krebs \cite{Cromme2017DiscreteAB}
提出一种自由节点一阶样条算法,
可在有限步骤内获得离散实值函数的最佳全局最小二乘近似,
并通过医学应用(如最小杀菌浓度 MBC、最小抑菌浓度 MIC)验证其有效性。

平滑样条(对应定义~\ref{def:cubicSmoothingFunctionsFitting} 和
\ref{def:smoothingFunctionsFitting})
的提出可追溯至 Schoenberg \cite{Schoenberg1964},
其研究基于 Whittaker \cite{Whittaker1922}提出的平滑思想。
平滑样条最初在数值分析领域发展,
Wahba \cite{Wahba1990}等人后续研究揭示了其统计学特性。
因此,其开始广泛应用于数据分析和各应用领域。
在仿真领域,de Boer、Nicola 和 Rubinstein \cite{deBoer2000AdaptiveIS}
利用三次平滑样条估计稀有事件概率中的转移概率函数;
Keys 和 Rees \cite{Keys2004ASM}则通过
薄板平滑样条(thin-plate smoothing splines)
提出了一种序贯仿真优化策略;而 Pedro 和 Isabel \cite{Santos2010}
将平滑样条用于非参数化元建模。

因此,我们希望掌握 DLS 样条逼近和平滑样条拟合的基础理论,
并进行相应的编程实现。

\section{国内外研究现状}

\subsection{研究方向及进展}

\subsubsection{ DLS 样条逼近}
关于定义~\ref{def:DiscreteLeastSquaresFunctionsApproximation} 中的问题,
经典教材\cite{GuideToSplines}指出,可以使用正规方程组解决此问题,
最终得到一个 B 格式的样条函数。

具体地,假设一个递增节点组$\mathbf{t}=\{t_{i}\}_{i=1}^{n+k}$和对应的一组$k$阶
B 样条基函数$\{B_{i}\}_{i=2}^{n+1}$。
则定义~\ref{def:DiscreteLeastSquaresFunctionsApproximation} 中的$X$可以定义为
$X:=\text{span}\{B_{2},B_{3},\cdots,B_{n+1}\}$。
并且,我们可以假设待求的样条$f=\sum_{j=1}^{n}a_{j}B_{j+1}$。
接着,我们考虑一个合适的内积:
\begin{defn}[离散内积]
  \label{def:DiscreteInnerProduct}
  对于有限个互不相同的数据点$x_{1},x_{2},\cdots,x_{N}$,
  定义$\mathbb{R}$上的离散内积
  \begin{equation}
    \label{eq:DiscreteInnerProduct}
    \langle u(x),v(x)\rangle
    =\sum_{i=1}^{N}w_{i}u(x_{i})v(x_{i}),
  \end{equation}
  其中函数$u,v:\mathbb{R}\to \mathbb{R}$,$w_{i}>0$为$x_{i}$处的权重。
\end{defn}

通过这个定义,结合 Gram 矩阵的相关知识,我们可以得到
关于$\mathbf{a}=[a_{1},a_{2},\cdots,a_{n}]^{T}$的矩阵方程:
\begin{equation}
  \label{eq:DLSSplinesApproximationviaNormalEquations}
  G(B_{2},B_{3},\cdots,B_{n+1})^{T}\mathbf{a}=\mathbf{c},
\end{equation}
其中$G(B_{2},B_{3},\cdots,B_{n+1})=
\left(\left\langle B_{i+1}, B_{j+1}\right\rangle\right)_{n\times n}$
为 Gram 矩阵,
$\mathbf{c}:=[\langle y,B_{2}\rangle,
\langle y,B_{3}\rangle,\cdots,\langle y,B_{n+1}\rangle]^{T}$,
函数$y:\mathbb{R}\to \mathbb{R}$满足$y(x_{i})=y_{i},\ i=1,2,\cdots,N$,
内积$\langle\cdot,\cdot\rangle$
由~(\ref{eq:DiscreteInnerProduct})定义。

记$A:=G(B_{2},B_{3},\cdots,B_{n+1})^{T}$。
由 B 样条的定义知,当$|i-j|\ge k$时,
$B_{i}B_{j}\equiv 0$,从而$\langle B_{i},B_{j}\rangle=0$。
因此$A$的形式类似为(以$k=4,n=10$为例)
\begin{displaymath}
  \begin{aligned}
    \begin{pmatrix}
        \ast& \ast& \ast& \ast&     &     &     &     &     &     \\
        \ast& \ast& \ast& \ast& \ast&     &     &     &     &     \\
        \ast& \ast& \ast& \ast& \ast& \ast&     &     &     &     \\
        \ast& \ast& \ast& \ast& \ast& \ast& \ast&     &     &     \\
            & \ast& \ast& \ast& \ast& \ast& \ast& \ast&     &     \\
            &     & \ast& \ast& \ast& \ast& \ast& \ast& \ast&     \\
            &     &     & \ast& \ast& \ast& \ast& \ast& \ast& \ast\\
            &     &     &     & \ast& \ast& \ast& \ast& \ast& \ast\\
            &     &     &     &     & \ast& \ast& \ast& \ast& \ast\\
            &     &     &     &     &     & \ast& \ast& \ast& \ast
    \end{pmatrix}\\
  \end{aligned}
  ,
\end{displaymath}
在$k$固定,$n$很大时是一个对称稀疏矩阵。进一步,它还是一个正定矩阵。
因此,我们可以对$A$进行$LDL^{T}$分解,
从而只需依次解决三个容易的矩阵方程:
$L\mathbf{a}_{0}=\mathbf{c}$,$D\mathbf{a}_{1}=\mathbf{a}_{0}$,
$L^{T}\mathbf{a}=\mathbf{a}_{1}$。如此,我们便得到了$f$的表达式。

\subsubsection{三次平滑样条拟合}
对于定义~\ref{def:cubicSmoothingFunctionsFitting} 中的问题,
针对$\lambda\equiv 1$的情况,$X=\mathbb{S}_{3}^{2}(\mathbf{x})$。
经典教材\cite{GuideToSplines}指出,
样条函数$f\in X$是存在唯一的,且满足自然边界条件
(即$f''(x_{1})=f''(x_{N})=0$)。
它是 pp 格式的样条函数。
相似的计算方法在 Reinsch
的文献\cite{SmoothingBySplineFunctions}亦有体现。

此方法的核心是,我们需要求解以下的线性方程组:
\begin{equation}
  \label{eq:cubicSmoothingSplineEquation}
  (pR+6(1-p)Q^{T}D^{2}Q)\mathbf{u}=Q^{T}\mathbf{y},
\end{equation}
其中
\begin{align*}
  D&=\text{diag}\left(w_{1}^{-\frac{1}{2}},w_{2}^{-\frac{1}{2}},\cdots,
  w_{N}^{-\frac{1}{2}}\right),\\
  R&=\left(
  \begin{array}{ccccc}
	2(h_1+h_2)& h_2&& &\\
	h_2& 2(h_2+h_3)& h_3&&\\
	& h_3  &2(h_3+h_4)  &\ddots&\\
	&&\ddots&\ddots&h_{N-2}\\
	&&& h_{N-2} &2(h_{N-2}+h_{N-1})
  \end{array}
  \right)\in \mathbb{R}^{(N-2)\times(N-2)},\\
   Q&=\left(
    \begin{array}{cccc}
      \frac{1}{h_1}& & &\\
      -\frac{1}{h_1}-\frac{1}{h_2}& \frac{1}{h_{2}}& &\\
      \frac{1}{h_2}& -\frac{1}{h_2}-\frac{1}{h_3}  &\ddots  &\\
                   &\frac{1}{h_{3}}&\ddots&\frac{1}{h_{N-2}}\\
                   &&\ddots&-\frac{1}{h_{N-2}}-\frac{1}{h_{N-1}}\\
      &&&\frac{1}{h_{N-1}}
    \end{array}
      \right)\in \mathbb{R}^{N\times (N-2)},\\
  h_{i}&=x_{i+1}-x_{i},\ i=1,2,\cdots,N-1.
\end{align*}

记系数矩阵为$A$,由$D,\ Q$和$R$的结构知,$A$是一个对称正定的五对角矩阵。
因此,可以对$A$使用$LDL^{T}$分解,其中单位下三角矩阵$L$的带宽为2。

\subsubsection{平滑样条拟合}
对于定义~\ref{def:smoothingFunctionsFitting} 中的问题,
针对$\lambda\equiv 1$的情况,
Reinsch \cite{SmoothingBySplineFunctions}对
三次自然边界条件下 pp 格式的样条给出了一套解决方案。
其主要思想是,
相比于定义~\ref{def:cubicSmoothingFunctionsFitting} 中的问题而言,
只需计算使得$E(f)=S$成立的$\rho$的值。因为假设已知$\rho$的值,
只需找到$f$,使如下式子成立:
\begin{equation}
  \label{eq:SmoothingSplineEquation}
  f=\arg \min_{g\in X}\left(\rho E(g)+ F(D^{2}g)\right).
\end{equation}
注意到若令~(\ref{eq:cubicSmoothingSpline})中的$p=\rho/(\rho+1)$,
则
求解~(\ref{eq:SmoothingSplineEquation})中$f$的方法和
求解~(\ref{eq:cubicSmoothingSpline})中$f$的方法一致)。
而计算$\rho$的方法在 Reinsch 的文献
\cite{SmoothingBySplineFunctions,SmoothingBySplineFunctionsII}
中有所提及,其主要思想是使用 Newton 法求解一个关于$\rho$的方程:
\begin{equation}
  \label{eq:rhoEqnForNewton1}
  g(\rho):= \frac{1}{E_{\rho}^{\frac{1}{2}}}
  -\frac{1}{S^{\frac{1}{2}}}=0,
\end{equation}
其中$E_{\rho}=E(f)$是一个关于$\rho$的函数(因为$f$的求解与$\rho$有关)。
这个方程虽然等价于$E(f)=S$,
但因为$g(\rho)$的图像
是上凹的,且随着$\rho$的增大会逐渐变得线性,因此
用 Newton 法求解$g(\rho)=0$会收敛,且收敛得会很快。

\subsection{存在问题}

\subsubsection{ DLS 样条逼近}
虽然正规方程组的解法在理论上可行,但当求解的规模越來越大时,
系数矩阵的条件数会平方增长,这对求解精度是不利的。

\subsubsection{三次平滑样条拟合}
相关文献\cite{GuideToSplines,SmoothingBySplineFunctions}中
均是对$\lambda\equiv 1$情形的解法表述,并没有提及
$\lambda$不恒等于1时,样条函数$f$所在的空间,以及求解$f$的方法。

\subsubsection{平滑样条拟合}
目前,对于任意阶样条的情形仍不清楚。我们需要确定,在自然边界条件
(即$D^{j}f(x_{1})=D^{j}f(x_{N})=0,\ j=m,m+1,\cdots,2m-2$)下,
样条函数$f$是否存在且唯一。其次,由于样条可以是任意阶的,所以
使用 pp 样条可能不如使用 B 样条方便。

总而言之,对于同一组平面上的点,针对不同的拟合目标,会存在不同的解法,
从而得到的样条函数不尽相同,会影响最后拟合出的曲线效果。

\section{研究展望}
对于 DLS 样条逼近问题,我们可以考虑使用 QR 分解的方法进行求解。
其好处是,求解规模变大时,系数矩阵的条件数线性增长,相比于
正规方程组而言应当有较高的求解精度。具体而言,
我们可以仿照 B 样条插值问题所对应的方程,
类似地在拟合问题中能够得到一个超定方程组,其中系数矩阵为
%样条配置矩阵(spline collocation matrix)。该矩阵是
几乎块对角矩阵(almost block-diagonal matrix),
可以对其进行块 QR 分解(block QR factorization)
\cite{Boor1976SOLVEBLOKAP},从而在最小二乘意义下求出方程组的解。

对于三次平滑样条拟合,我们希望解决$\lambda$不恒等于1的情况。
这可能需要我们确定一个新的函数空间,并且需要重新修改相关的理论。

对于$2m$阶平滑样条拟合,我们可能需要以 B 样条基函数的形式来确定函数空间,
并且需要制定新的理论,比如参考 de Boor 的论文\cite{Boor2001CALCULATIONOT}。

\newpage
\begingroup
    \linespreadsingle{}
    \printbibliography[title={参考文献}]
\endgroup
