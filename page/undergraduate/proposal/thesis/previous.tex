{
    \setlength{\parindent}{0em}
    \par {\zihao{4}\bfseries 一、题目:\Title}
    \\
    \par {\zihao{4}\bfseries 二、指导教师对文献综述、开题报告、外文翻译的具体要求:}

    \qquad\textbf{文献综述要求:}
    根据阅读的国内外文献撰写文献综述报告,要求根据主题展开,文献综述内容要切题,包括

    \qquad(1)简述曲线拟合领域经典理论方法以及近几年的进展,掌握当前该领域的研究前沿,
    分析你毕业论文研究的内容与这些论文的差异和相关。
    应包括基础的曲线拟合理论,基于样条函数进行曲线拟合的方法,及前述内容的编程实现。

    \qquad(2)从内积空间正交系统的角度叙述经典的最小二乘曲线拟合方法,
    对最小二乘样条函数拟合和平滑样条函数拟合进行介绍,并分析不同方法对拟合结果的影响。

    \qquad(3)简述各参考文献的创新性、存在的问题或未能解决的问题。

    \qquad(4)要求翻译其中的一篇外文文献,结构完整,语句通顺。
    可以是一篇经典论文或者经典专著(如A Practical Guide to Splines)中的一节。

    \qquad\textbf{开题报告要求:}

    \qquad(1)介绍曲线拟合的意义(包括理论意义和实际应用意义,并分析课题与本专业的关系)。
    简述基于样条函数进行曲线拟合相比于其他拟合方法的优势,分析不同样条拟合方法的关系和优劣势,
    以及如何设计合适的数据结构进行编程实现。

    \qquad(2)根据文献综述分析课题的研究背景
    (即要解决什么问题?本文所讨论问题的角度与已有参考文献中所涉及的问题的差异,课题的主要创新点是什么?)
    强调 MATLAB 软件被禁用对我国计算软件的影响以及发展自主产权计算软件的迫切性。

    \qquad(3)选题的可行性分析(从北太天元的软件架构以及曲线拟合函数的实现思路等方面去说明)。

    \qquad(4)主要研究内容(这部分要展开写,主要包括理论研究内容以及拟实现的曲线拟合函数的契约等)。

    \qquad(5)根据研究的内容写出具体的实施计划。

    \qquad(6)明确论文最后预期结果。
}

\mbox{} \vfill

\signature{指导教师(签名)}
% comment the line above and uncomment the line below if you want to set a signature with a specific date.
% \signaturewithdate{指导教师(签名)}{1897}{5}{21}
